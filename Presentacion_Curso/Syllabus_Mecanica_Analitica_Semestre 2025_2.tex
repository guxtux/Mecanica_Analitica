\documentclass[hidelinks,12pt]{article}
\usepackage[left=0.25cm,top=1cm,right=0.25cm,bottom=1cm]{geometry}
%\usepackage[landscape]{geometry}
\textwidth = 20cm
\hoffset = -1cm
\usepackage[utf8]{inputenc}
\usepackage[spanish,es-tabla, es-lcroman]{babel}
\usepackage[autostyle,spanish=mexican]{csquotes}
\usepackage[tbtags]{amsmath}
\usepackage{nccmath}
\usepackage{amsthm}
\usepackage{amssymb}
\usepackage{mathrsfs}
\usepackage{graphicx}
\usepackage{subfig}
\usepackage{caption}
%\usepackage{subcaption}
\usepackage{standalone}
\usepackage[outdir=./Imagenes/]{epstopdf}
\usepackage{siunitx}
\usepackage{physics}
\usepackage{color}
\usepackage{float}
\usepackage{hyperref}
\usepackage{multicol}
\usepackage{multirow}
%\usepackage{milista}
\usepackage{anyfontsize}
\usepackage{anysize}
%\usepackage{enumerate}
\usepackage[shortlabels]{enumitem}
\usepackage{capt-of}
\usepackage{bm}
\usepackage{mdframed}
\usepackage{relsize}
\usepackage{placeins}
\usepackage{empheq}
\usepackage{cancel}
\usepackage{pdfpages}
\usepackage{wrapfig}
\usepackage[flushleft]{threeparttable}
\usepackage{makecell}
\usepackage{fancyhdr}
\usepackage{tikz}
\usepackage{bigints}
\usepackage{menukeys}
\usepackage{tcolorbox}
\tcbuselibrary{breakable}
\usepackage{scalerel}
\usepackage{pgfplots}
\usepackage{pdflscape}
\pgfplotsset{compat=1.16}
\spanishdecimal{.}

\AtBeginDocument{\RenewCommandCopy\qty\SI}
\ExplSyntaxOn
\msg_redirect_name:nnn { siunitx } { physics-pkg } { none }
\ExplSyntaxOff

\renewcommand{\baselinestretch}{1.5} 
\renewcommand\labelenumii{\theenumi.{\arabic{enumii}})}

\newcommand{\python}{\texttt{python}}
\newcommand{\textoazul}[1]{\textcolor{blue}{#1}}
\newcommand{\azulfuerte}[1]{\textcolor{blue}{\textbf{#1}}}
\newcommand{\funcionazul}[1]{\textcolor{blue}{\textbf{\texttt{#1}}}}

\newcommand{\pderivada}[1]{\ensuremath{{#1}^{\prime}}}
\newcommand{\sderivada}[1]{\ensuremath{{#1}^{\prime \prime}}}
\newcommand{\tderivada}[1]{\ensuremath{{#1}^{\prime \prime \prime}}}
\newcommand{\nderivada}[2]{\ensuremath{{#1}^{(#2)}}}


\newtheorem{defi}{{\it Definición}}[section]
\newtheorem{teo}{{\it Teorema}}[section]
\newtheorem{ejemplo}{{\it Ejemplo}}[section]
\newtheorem{propiedad}{{\it Propiedad}}[section]
\newtheorem{lema}{{\it Lema}}[section]
\newtheorem{cor}{Corolario}
\newtheorem{ejer}{Ejercicio}[section]

\newlist{milista}{enumerate}{2}
\setlist[milista,1]{label=\arabic*)}
\setlist[milista,2]{label=\arabic{milistai}.\arabic*)}
\newlength{\depthofsumsign}
\setlength{\depthofsumsign}{\depthof{$\sum$}}
\newcommand{\nsum}[1][1.4]{% only for \displaystyle
    \mathop{%
        \raisebox
            {-#1\depthofsumsign+1\depthofsumsign}
            {\scalebox
                {#1}
                {$\displaystyle\sum$}%
            }
    }
}
\def\scaleint#1{\vcenter{\hbox{\scaleto[3ex]{\displaystyle\int}{#1}}}}
\def\scaleoint#1{\vcenter{\hbox{\scaleto[3ex]{\displaystyle\oint}{#1}}}}
\def\scaleiiint#1{\vcenter{\hbox{\scaleto[3ex]{\displaystyle\iiint}{#1}}}}
\def\bs{\mkern-12mu}

\newcommand{\Cancel}[2][black]{{\color{#1}\cancel{\color{black}#2}}}

\author{M. en C. Gustavo Contreras Mayén. \texttt{curso.fisica.comp@gmail.com}\\
Alfredo Rodríguez González \texttt{alfredojo1997@ciencias.unam.mx}}
\title{Syllabus del Curso de Mecánica Analítica \\ {\large Semestre 2025-2 Grupo 8391}}
\date{ }
% \makeatletter
% \renewcommand{\@biblabel}[1]{}
% \renewenvironment{thebibliography}[1]
%      {\section*{\refname}%
%       \@mkboth{\MakeUppercase\refname}{\MakeUppercase\refname}%
%       \list{}%
%            {\labelwidth=0pt
%             \labelsep=0pt
%             \leftmargin1.5em
%             \itemindent=-1.5em
%             \advance\leftmargin\labelsep
%             \@openbib@code
%             }%
%       \sloppy
%       \clubpenalty4000
%       \@clubpenalty \clubpenalty
%       \widowpenalty4000%
%       \sfcode`\.\@m}
% \makeatother

\usepackage[backend=biber, style=ieee, sorting=ynt]{biblatex}
\addbibresource{LibrosMecanica2025.bib}


\begin{document}

\renewcommand\labelenumii{\theenumi.{\arabic{enumii}}}
\maketitle
\fontsize{12}{12}\selectfont

\textbf{Lugar: } Salón 0129.
\par
\textbf{Horario: } Lunes, Miércoles y Viernes de 16 a 18 pm.
\par
\par
\section{Objetivos y Temario:}

Se trabajará el temario oficial de la asignatura, que está disponible en:

\href{https://www.fciencias.unam.mx/sites/default/files/temario/611.pdf}{https://www.fciencias.unam.mx/sites/default/files/temario/611.pdf}

Los objetivos propuestos son que el alumno:
\begin{enumerate}
    \item Maneje las leyes de la mecánica con un nivel más alto de matemáticas, con el formalismo de las ecucaciones diferenciales
    \item Aprenderá las formulaciones de Lagrange y Hamilton.
    \item Abordará el estudio de sistemas no lineales.
    \item Así como de la teoría de perturbaciones.
\end{enumerate}
Aunque no forma parte del contenido del curso, es conveniente conocer y/o manejar lenguajes de programación para resolver ecuaciones de manera numérica, o programas tipo Matlab, Maple, Derive, Mathematica, para la revisión de ciertos casos y seguimiento con gráficas.

\section{Metodología de Enseñanza.}

\noindent
\textbf{Antes de la clase.}
\par
Para facilitar la discusión en el aula, el alumno revisará el material de trabajo que se le proporcionará oportunamente, así como la solución de algunos ejercicios, de tal manera que llegará a la clase conociendo el tema a desarrollar. Daremos por entendido de que el alumno realizará la lectura y actividades establecidas.
\par
\textbf{Durante la clase.}
\par
Habrá exposición con dialógo por parte del equipo académico, se resolverán ejercicios en el pizarrón, se revisarán los ejercicios que se dejaron para resolver en clase, se discutirán los temas y se resolverán dudas.
\par
Se espera que el alumno participe activamente en la clase, ya que es un espacio para aclarar dudas y reforzar los temas.
\par
\textbf{Después de la clase.}
\par
Al concluir la clase, se tendrán ejercicios a resolver, para que pueda repasar el tema visto en clase. En caso de que algún ejercicio haya quedado incompleto.

\section{Evaluación.}

Los elementos y la proporción de la calificación total del curso, se distribuyen de la siguiente manera:
\begin{enumerate}[label=\alph*)]
\item \textbf{Tareas $\mathbf{40\%}$:} Se tendrán ejercicios por cada uno de los siete temas del curso. Se entregarán los enunciados oportunamente para que cuenten con el tiempo para resolverlos y entregarlos. No se recibirán tareas de manera extermporánea o enviadas por correo electrónico.
\item \textbf{Exámenes $\mathbf{60\%}$} : Se tendrán tres exámenes parciales durante el curso.
\par
Un examen se considera acreditado cuando la calificación obtenida es mayor o igual a seis. En caso de que en algún (o más) examen(es) la calificación sea menor a seis, el alumno ya es candidato a presentar el examen final del curso.
\par
No se tendrán reposiciones de los exámenes parciales.
\end{enumerate}
La calificación final del curso se obtendrá de las calificaciones de cada uno de los componentes de la evaluación: tareas y exámenes. En el caso de obtener una calificación mayor o igual a $6$, será la que se asentará en el acta del curso.

\section{Examen final.}

Para presentar el examen final del curso se deben de cumplir cada una de las siguientes condiciones:
\begin{enumerate}\label{ref:criterios_final}
\item Que en un examen (o más), la calificación sea menor a seis. Si en los exámenes se tiene una calificación aprobatoria y el promedio final de la asignatura es aprobatorio, no se permite presentar el examen final para \enquote{subir} la calificación del curso.
\item Se hayan presentado y entregado los tres exámenes parciales.
\end{enumerate}
En caso de que no se cumplan las condiciones anteriores, no se podrá presentar el examen final. De acuerdo al Reglamento de Estudios Profesionales, habrá dos oportunidades para presentar el examen final, cuyas fechas se indican en el calendario del semestre 2025-2.
\par
Puntalizando sobre el examen final:
\begin{enumerate}[label=\roman*)]
\item Si en la primera ronda de examen final, la calificación obtenida es aprobatoria (mayor o igual a seis), ésta es la que se asentará en el acta del curso, ya no se promedia con los otros elementos de evaluación.
\item Si la calificación del examen final en la primera ronda es no aprobatoria, se aplicará nuevamente un examen final en la segunda ronda. La calificación obtenida en esta segunda ronda, es la que se asentará en el acta del curso.
\item Si el alumno no se presenta a la primera ronda del examen final, tendrá cinco como calificación final. Ya no podrá presentar la segunda ronda del examen final.
\end{enumerate}
\par
\textbf{Importante: } \emph{En caso de haber presentado al menos un examen y/o haber entregado al menos una tarea}, en el caso de que ya no se tenga un posterior registro de entregas, se considerará que abandonaron el curso, al no cumplir con los puntos de la lista del numeral \ref{ref:criterios_final}, no se podrá presentar el examen final del curso.
\par
Se asentará en el acta de calificaciones \textcolor{blue}{No Presentó (NP)}, si y solo si: el alumno no entrega tarea alguna y no entrega algún examen (¿?). Ocupando nuevamente el Reglamento de Estudios Profesionales, tomen en cuenta que:
\begin{itemize}
\setlength\itemsep{1pt}
\item No \enquote{se guardan calificaciones}.
\item No se renuncia a una calificación.
\end{itemize}

\section{Temario.}

Los temas a revisar durante el curso son:
\begin{enumerate}
\item Dinámica de una partícula.
\begin{enumerate}
    \item Leyes de Newton.
    \item Fuerzas.
    \item Energía cinética y potencial.
    \item Movimiento en una dimensión. Espacio fase, Movimiento armónico y anarmónico.
    \item Movimiento en dos y tres dimensiones. Teoría de perturbaciones.
\end{enumerate}
\item Campo central.
\begin{enumerate}
    \item Teoremas de conservación.
    \item El problema de Kepler.
    \item Órbitas elípticas.
    \item Satélites.
\end{enumerate}
\item Sistemas de partículas.
\begin{enumerate}
    \item Centro de masa y principios de conservación.
    \item El problema de dos cuerpos.
    \item Colisiones y dispersión.
    \item El problema de los tres cuerpos.
\end{enumerate}
\item Teorías de Lagrange y de Hamilton.
\begin{enumerate}
    \item Coordenadas generalizadas y el principio de Hamilton.
    \item Ecuaciones de Euler-Lagrange.
    \item Principios de conservación y coordenadas ignorables.
    \item Ecuaciones de Hamilton.
\end{enumerate}
\item Sistemas de referencia.
\begin{enumerate}
    \item Sistemas acelerados.
    \item Coriolis.
\end{enumerate}
\item Cuerpo rígido.
\begin{enumerate}
    \item Rotación y efecto giroscópico.
    \item Momento de inercia y los ángulos de Euler.
    \item El trompo simétrico.
\end{enumerate}
\item Vibraciones pequeñas.
\begin{enumerate}
    \item Osciladores acoplados.
    \item Modos normales.
\end{enumerate}
\end{enumerate}

\section{Fechas importantes.}

\begin{itemize}
\setlength\itemsep{1pt}
\item Lunes 27 de enero. Inicio del semestre 2025-2.
\item \textcolor{red}{Lunes 3 de febrero. Día feriado.}
\item \textcolor{red}{Lunes 17 de marzo. Día feriado.}
\item \textcolor{red}{Lunes 14 al viernes 18 de abril. Semana Santa.}
\item Viernes 23 de mayo. Termian el semestre 2025-2.
\item Del lunes 26 de mayo al viernes 30 de mayo, primera semana de finales.
\item Del lunes 2 al viernes 6 de junio, segunda semana de finales.
\end{itemize}

\section{Bibliografía.}

Se recomienda la consulta de los siguientes textos, en cada uno de los temas se propocionará bibliografía adicional para una mejor comprensión del tema.
\nocite{*}
% \printbibliography[keyword={computacional}, title={Referencias Física Computacional}]
% \printbibliography[keyword={python}, title={Referencias pyhton, matplolib, jupyter}]
\printbibliography

\end{document}