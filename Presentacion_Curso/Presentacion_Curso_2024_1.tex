\documentclass[12pt]{beamer}
\usepackage{../Estilos/BeamerFC}
\usepackage{../Estilos/ColoresLatex}
\usefonttheme{serif}
\usetheme{Warsaw}
\usecolortheme{seahorse}
%\useoutertheme{default}
\setbeamercovered{invisible}
% or whatever (possibly just delete it)
\setbeamertemplate{section in toc}[sections numbered]
\setbeamertemplate{subsection in toc}[subsections numbered]
\setbeamertemplate{subsection in toc}{\leavevmode\leftskip=3.2em\rlap{\hskip-2em\inserttocsectionnumber.\inserttocsubsectionnumber}\inserttocsubsection\par}
\setbeamercolor{section in toc}{fg=blue}
\setbeamercolor{subsection in toc}{fg=blue}
\setbeamercolor{frametitle}{fg=blue}
\setbeamertemplate{caption}[numbered]

\setbeamertemplate{footline}
\beamertemplatenavigationsymbolsempty
\setbeamertemplate{headline}{}


\makeatletter
\setbeamercolor{section in foot}{bg=gray!30, fg=black!90!orange}
\setbeamercolor{subsection in foot}{bg=blue!30}
\setbeamercolor{date in foot}{bg=black}
\setbeamertemplate{footline}
{
  \leavevmode%
  \hbox{%
  \begin{beamercolorbox}[wd=.333333\paperwidth,ht=2.25ex,dp=1ex,center]{section in foot}%
    \usebeamerfont{section in foot} \insertsection
  \end{beamercolorbox}%
  \begin{beamercolorbox}[wd=.333333\paperwidth,ht=2.25ex,dp=1ex,center]{subsection in foot}%
    \usebeamerfont{subsection in foot}  \insertsubsection
  \end{beamercolorbox}%
  \begin{beamercolorbox}[wd=.333333\paperwidth,ht=2.25ex,dp=1ex,right]{date in head/foot}%
    \usebeamerfont{date in head/foot} \insertshortdate{} \hspace*{2em}
    \insertframenumber{} / \inserttotalframenumber \hspace*{2ex} 
  \end{beamercolorbox}}%
  \vskip0pt%
}
\makeatother

\makeatletter
\patchcmd{\beamer@sectionintoc}{\vskip1.5em}{\vskip0.8em}{}{}
\makeatother

%\newlength{\depthofsumsign}
%\setlength{\depthofsumsign}{\depthof{$\sum$}}
% \newcommand{\nsum}[1][1.4]{% only for \displaystyle
%     \mathop{%
%         \raisebox
%             {-#1\depthofsumsign+1\depthofsumsign}
%             {\scalebox
%                 {#1}
%                 {$\displaystyle\sum$}%
%             }
%     }
% }
\def\scaleint#1{\vcenter{\hbox{\scaleto[3ex]{\displaystyle\int}{#1}}}}
\def\scaleoint#1{\vcenter{\hbox{\scaleto[3ex]{\displaystyle\oint}{#1}}}}
\def\bs{\mkern-12mu}


\date{14 de agosto de 2023}
\title{Curso de Física Computacional}
\subtitle{Semestre 2024-1}


\newcommand\RBox[1]{%
  \tikz\node[draw,rounded corners,align=center,] {#1};%
}


\begin{document}
\fontsize{14}{14}\selectfont
\spanishdecimal{.}
\maketitle

\section*{Contenido}
\frame{\tableofcontents[currentsection, hideallsubsections]}

\begin{frame}
\frametitle{Equipo académico}
\begin{center}
\RBox{
M. en C. Gustavo Contreras Mayén \\
\href{mailto:gux7avo@ciencias.unam.mx}{gux7avo@ciencias.unam.mx}
}
\vskip 1cm
\RBox{
M. en C. Abraham Lima Buendía \\
\href{mailto:abraham3081@ciencias.unam.mx}{abraham3081@ciencias.unam.mx}
}
\end{center}
\end{frame}

\section{Presentación del curso}
\frame{\tableofcontents[currentsection, hideothersubsections]}
\subsection{Objetivos}

\begin{frame}
\frametitle{Objetivos 1}
\setbeamercolor{item projected}{bg=red,fg=white}
\setbeamertemplate{enumerate items}{%
\usebeamercolor[bg]{item projected}%
\raisebox{1.5pt}{\colorbox{bg}{\color{fg}\footnotesize\insertenumlabel}}%
}
\begin{enumerate}[<+->]
\item  El propósito del curso es enseñar al estudiante las ideas de computabilidad usadas en distintas áreas de la física para resolver un conjunto de problemas modelo. 
\seti
\end{enumerate}
\end{frame}
\begin{frame}
\frametitle{Objetivos 1}    
A partir de planteamientos analíticos se pretende obtener resultados numéricos reproducibles consistentes, y que predigan situaciones físicas asociadas al problema bajo estudio.
\end{frame}
\begin{frame}
\frametitle{Objetivos 2}
\setbeamercolor{item projected}{bg=red,fg=white}
\setbeamertemplate{enumerate items}{%
\usebeamercolor[bg]{item projected}%
\raisebox{1.5pt}{\colorbox{bg}{\color{fg}\footnotesize\insertenumlabel}}%
}
\begin{enumerate}[<+->]
\conti
\item El alumno debe asimilar las ideas básicas del análisis numérico, como son las de estabilidad en el cálculo y la sensibilidad de las respuestas a las perturbaciones en la estructura del problema.
\seti
\end{enumerate}
\end{frame}
\begin{frame}
\frametitle{Objetivos 3}
\setbeamercolor{item projected}{bg=red,fg=white}
\setbeamertemplate{enumerate items}{%
\usebeamercolor[bg]{item projected}%
\raisebox{1.5pt}{\colorbox{bg}{\color{fg}\footnotesize\insertenumlabel}}%
}
\begin{enumerate}[<+->]
\conti
\item El curso también le dará al estudiante capacidad de juicio sobre la calidad de los resultados numéricos obtenidos.
\seti
\end{enumerate}
\end{frame}
\begin{frame}
\frametitle{Objetivos 4}
\setbeamercolor{item projected}{bg=red,fg=white}
\setbeamertemplate{enumerate items}{%
\usebeamercolor[bg]{item projected}%
\raisebox{1.5pt}{\colorbox{bg}{\color{fg}\footnotesize\insertenumlabel}}%
}
\begin{enumerate}[<+->]
\conti
\item En particular se hará énfasis en la confiabilidad de los resultados respecto a los errores tanto del algoritmo de solución como de las limitaciones numéricas de la computadora. 
\conti
\end{enumerate}
\end{frame}
\begin{frame}
\frametitle{Objetivos 4}
Esta capacidad se adquirirá a lo largo del curso comparando resultados numéricos con otros tipos de análisis, en las regiones en las cuales se pueden llevar ambos a cabo.
\end{frame}
\begin{frame}
\frametitle{Objetivos 5}
\setbeamercolor{item projected}{bg=red,fg=white}
\setbeamertemplate{enumerate items}{%
\usebeamercolor[bg]{item projected}%
\raisebox{1.5pt}{\colorbox{bg}{\color{fg}\footnotesize\insertenumlabel}}%
}
\begin{enumerate}[<+->]
\conti
\item  Por otra parte permitirá al estudiante explorar regiones de comportamiento físico sólo accesibles al cálculo numérico.
\end{enumerate}
\end{frame}

\section{Sobre el curso}
\frame{\tableofcontents[currentsection, hideothersubsections]}
\subsection{Lugar y horario}

\begin{frame}
\frametitle{Lugar y horario} 
\textbf{Lugar: } Laboratorio de Enseñanza en Cómputo. Tlahuizcalpan.
\\
\bigskip
\textbf{Horario: } Lunes y Miércoles de 15 a 18.
\end{frame}

\subsection{Metodología de Enseñanza}

\begin{frame}
\frametitle{Metodología de Enseñanza - 1}
\textbf{Antes de la clase.}
\\
\vspace{0.5em}
Para facilitar la discusión en el aula, el alumno revisará el material de trabajo que se le proporcionará oportunamente, así como la solución de algunos ejercicios, de tal manera que llegará a la clase conociendo el tema a desarrollar.
\end{frame}
\begin{frame}
\frametitle{Metodología de Enseñanza - 1}
\textbf{Antes de la clase.}
\\
\vspace{0.5em}
Daremos por entendido de que el alumno realizará la lectura y actividades establecidas.
\\
\bigskip
\pause
\begin{alertblock}{Aviso importante}
Daremos por entendido de que el alumno realizará la lectura y/o actividades.
\end{alertblock}
\end{frame}
\begin{frame} 
\frametitle{Metodología de Enseñanza - 2}
\textbf{Durante la clase.}
\\
\vspace{0.5em}
Se dará un tiempo para la exposición con diálogo y discusión del material de trabajo con los temas a cubrir durante el semestre.
\end{frame}
\begin{frame} 
\frametitle{Metodología de Enseñanza - 2}
\textbf{Durante la clase.}
\\
\vspace{0.5em}
Se busca que sea un curso práctico por lo que se va a trabajar con los equipos de cómputo del laboratorio, de tal manera que habrá ejercicios para desarrollar durante la clase.
\end{frame}
\begin{frame} 
\frametitle{Metodología de Enseñanza - 2}
\textbf{Durante la clase.}
\\
\vspace{0.5em}
Un curso de este tipo requiere que el alumno resuelva problemas mediante un \enquote{pensamiento computacional}, que involucra las tareas de:
\pause
\setbeamercolor{item projected}{bg=cadmiumgreen,fg=cadmiumyellow}
\setbeamertemplate{enumerate items}{%
\usebeamercolor[bg]{item projected}%
\raisebox{1.5pt}{\colorbox{bg}{\color{fg}\footnotesize\insertenumlabel}}%
}
\begin{enumerate}[<+->]
\item Descomposición.
\item Reconocimiento de patrones.
\item Abstracción.
\item Pensamiento algorítmico.
\end{enumerate}
\end{frame}
\begin{frame} 
\frametitle{Metodología de Enseñanza - 2}
\textbf{Durante la clase.}
\\
\medskip
Considere que el curso no está enfocado al desarrollo de habilidades y/o técnicas de programación con un lenguaje en particular. 
\\
\bigskip
\pause
En un primer momento se revisará un planteamiento general de la solución, para posteriormente, implementarlo en la sintaxis del lenguaje \textoazul{\python}.
\end{frame}
\begin{frame} 
\frametitle{Metodología de Enseñanza - 2}
\textbf{Durante la clase.}
\\
\vspace{0.5em}
Si cuentan con una experiencia previa en programación (con cualquier lenguaje), será conveniente para el trabajo en clase.
\\
\bigskip
\pause
Si no han programado, se verán en la necesidad de dedicarle más tiempo tanto para revisar los materiales adicionales, así como para resolver los problemas y ejercicios.
\end{frame}
\begin{frame}
\frametitle{Herramienta de programación}
Será necesario utilizar una herramienta computacional para resolver ejercicios y problemas que se revisen en clase.
\\
\bigskip
Usaremos \textoazul{\python} dada su versatilidad y facilidad de manejo, así como \textoazul{\texttt{jupyter}} como \enquote{notebooks}, lo que nos facilitará el trabajo.
\end{frame}
\begin{frame}
\frametitle{Herramienta de programación}
Las técnicas de programación que vayan adquiriendo serán el reflejo de su trabajo fuera de clase.
\\
\bigskip
\pause
En caso de no trabajar o dedicarle el tiempo al curso, se complicará bastante, situación que esperamos no se presente.
\end{frame}
\begin{frame}
\frametitle{Guías adicionales de apoyo}
Se han elaborado guías de apoyo complementarias para la consulta tanto de los conceptos principales de la física involucrada en el problema, así como de programación con \textoazul{\python}.
\end{frame}
\begin{frame}
\frametitle{Guías adicionales de apoyo}
De esta manera tendrán una referencia adicional, por su cuenta deberán consultar otros materiales para complementar y conceptualizar el problema así como su solución.
\end{frame}

\subsection{¿Programación?}

\begin{frame}
\frametitle{¿Programación?}
Abordaremos la solución de un problema mediante el \textoazul{pensamiento computacional}, que conforme vayamos avanzando en el curso, nuestras habilidades se incrementarán, permitiendo entonces resolver los ejercicios cada vez, de mayor complejidad.
\end{frame}
\begin{frame}
\frametitle{¿Programación?}
El algoritmo que se proponga como solución, deberá \enquote{ejecutarse} por lo que debemos de revisar la solución, además de la congruencia de la misma con la física y sobre todo, el margen de error que devuelve la solución numérica.
\end{frame}
\begin{frame}
\frametitle{¿Programación?}
El curso de Física Computacional \textoazul{NO es un curso de programación bajo algún lenguaje en particular}.
\\
\bigskip
\pause
Es altamente recomendable que cuenten con conocimientos de programación básicos en algún lenguaje o software.
\end{frame}
\begin{frame}
\frametitle{Ya se programar!!}
Cuentan con la completa libertad de elegir el lenguaje o software para trabajar durante el curso:
\pause
\begin{multicols}{2}
\begin{itemize}
\item Fortran
\item Java
\item C++
\item C
\item Delphi
\item Mathematica
\item Maple
\item Matlab
\item Scilab
\item Octave
\end{itemize}
\end{multicols}
\end{frame}
\begin{frame}
\frametitle{Ya se programar!!}
Si es el caso que ya programen con algún otro lenguaje o software, deberán de entregar su código fuente y el archivo ejecutable.
\end{frame}
\begin{frame}
\frametitle{Software para el curso}
Usaremos dentro del curso la suite \textoazul{Anaconda}, que es de distribución libre y contiene una serie de herramientas y programas con lo que programar con \textoazul{\python} y \textoazul{\texttt{jupyter}}, será una tarea más sencilla.
\end{frame}
\begin{frame}
\frametitle{Anaconda}
La suite incluye un \emph{entorno de desarrollo}, terminales, sistema de debug y de consulta.
\\
\bigskip
Como es multiplataforma, se puede utilizar en entornos Linux, iOS y Windows. En los equipos del laboratorio tienen instalado Linux y la distribución es Fedora.
\end{frame}
\begin{frame}
\frametitle{Tema 0: Breve introducción a python.}
Con la finalidad de tener un mismo punto de partida en el curso, tendremos un repaso muy breve de programación básica con \python, este tema no será evaluado.
\end{frame}
\begin{frame}
\frametitle{Tema 0: Breve introducción a python.}
Tendremos con el Tema 0, un panorama general del uso del lenguaje, pero NO debemos de confiarnos y pensar que con esto, ya podremos programar con facilidad, mientras más práctica tengan, poco a poco mejorarán sus técnicas de programación.
\end{frame}
\begin{frame}
\frametitle{Opcionales}
Pueden traer una laptop para el trabajo en el curso, no es requisito, ya que tenemos equipos suficientes en el laboratorio.
\\
\medskip
\pause
Se recomienda que cuenten en sus equipos con el mismo software, las guías que hemos comentado, les brindarán la información para instalar los programas.
\end{frame}
\begin{frame}
\frametitle{Reglas dentro del aula de clase}
Habrá que seguir una serie de puntos tanto académicos como \enquote{administrativos} dentro del aula de clase para un buen desenvolvimiento y tener una clase amena. 
\end{frame}
\begin{frame}
\frametitle{Puntos académicos}
Consideren los siguientes puntos:
\setbeamercolor{item projected}{bg=cadetblue,fg=amber}
\setbeamertemplate{enumerate items}{%
\usebeamercolor[bg]{item projected}%
\raisebox{1.5pt}{\colorbox{bg}{\color{fg}\footnotesize\insertenumlabel}}%
}
\begin{enumerate}[<+->]
\item La clase inicia a las 3 pm.
\item Se utilizarán los equipos de cómputo del laboratorio.
\seti
\end{enumerate}
\end{frame}
\begin{frame}
\frametitle{Puntos académicos}
\setbeamercolor{item projected}{bg=cadetblue,fg=amber}
\setbeamertemplate{enumerate items}{%
\usebeamercolor[bg]{item projected}%
\raisebox{1.5pt}{\colorbox{bg}{\color{fg}\footnotesize\insertenumlabel}}%
}
\begin{enumerate}[<+->]
\conti    
\item Los equipos cuentan con conexión a internet, pero se espera que se utilice para atender la clase. \pause En caso de que se detecte una actividad ajena al curso (facebook, chats, etc.) se hará una primera llamada de aviso.
\item A la segunda llamada de aviso, se cancelará la evaluación correspondiente.
\end{enumerate}
\end{frame}
\begin{frame}
\frametitle{Puntos administrativos}
En lo que corresponde a la parte administrativa:
\setbeamercolor{item projected}{bg=cadetblue,fg=amber}
\setbeamertemplate{enumerate items}{%
\usebeamercolor[bg]{item projected}%
\raisebox{1.5pt}{\colorbox{bg}{\color{fg}\footnotesize\insertenumlabel}}%
}
\begin{enumerate}[<+->]
\item No se permite el consumo de alimentos y/o bebidas dentro del aula.
\item Se solicita dejen en modo silencioso el celular.
\item En caso de alguna eventualidad, se atenderán los protocolos de seguridad.
\end{enumerate}
\end{frame}
\begin{frame}
\frametitle{Trabajo durante la clase}
Los alumnos del curso contarán con una cuenta de acceso a los equipos.
\\
\bigskip
\pause
El uso y manejo de cada equipo pasa a ser responsabilidad de cada alumno. En caso de que algún periférico no funcione debidamente, se les pide lo reporten para que se solicite al área respectiva, su reparación o reemplazo.
\end{frame}
\begin{frame}
\frametitle{Metodología de Enseñanza - 3}
\textbf{Después de la clase.}
\\
\medskip
El curso \alert{requiere que le dediquen al menos el mismo número de horas de trabajo en casa}, es decir:
\pause
\begin{exampleblock}{Dedicación al curso}
Les va a demandar al menos seis horas de trabajo como mínimo.
\end{exampleblock}
\end{frame}
\begin{frame}
\frametitle{Metodología de Enseñanza - 3}
Si cuentan con una experiencia en programación, tienen un paso adelantado, pero si no han programado, se verán en la necesidad de dedicarle más tiempo.
\end{frame}
\begin{frame}
\frametitle{Metodología de Enseñanza - 3}
Con la finalidad de repasar lo que se vio en la clase, tendrán ejercicios que les permitan practicar nuevamente el tema.
\\
\bigskip
\pause
La entrega de los ejercicios de repaso contará también en el apartado de ejercicios para la evaluación.
\end{frame}

\section{Temario oficial}
\frame{\tableofcontents[currentsection, hideothersubsections]}
\subsection{Contenido del temario}

\begin{frame}
\frametitle{Temario del curso}
Llevaremos el temario oficial del curso, que está disponible en la página de la Facultad \href{https://www.fciencias.unam.mx/sites/default/files/temario/715.pdf}{- Temario -}, haciendo un ajuste en el orden de los temas, siendo entonces:
\end{frame}

\subsection*{Tema 1}

\begin{frame}
\frametitle{\textbf{Tema 1: Errores y artimética de punto flotante}}
\setbeamercolor{item projected}{bg=ceruleanblue,fg=amber}
\setbeamertemplate{enumerate items}{%
\usebeamercolor[bg]{item projected}%
\raisebox{1.5pt}{\colorbox{bg}{\color{fg}\footnotesize\insertenumlabel}}%
}
\begin{enumerate}[<+->]
\item Error absoluto y error relativo.
\item Precisión y exactitud.
\item Estabilidad y condicionamiento.
\item Artimética de punto flotante.
\end{enumerate}
\end{frame}

\subsection*{Tema 2}

\begin{frame}
\frametitle{\textbf{Tema 2: Operaciones matemáticas básicas}}
\setbeamercolor{item projected}{bg=ceruleanblue,fg=amber}
\setbeamertemplate{enumerate items}{%
\usebeamercolor[bg]{item projected}%
\raisebox{1.5pt}{\colorbox{bg}{\color{fg}\footnotesize\insertenumlabel}}%
}
\begin{enumerate}[<+->]
\item Interpolación y extrapolación.
\item Diferenciación numérica.
\item Integración numérica.
\end{enumerate}
\end{frame}

\subsection*{Tema 3}

\begin{frame}
\frametitle{\textbf{Tema 3: Ecuaciones diferenciales ordinarias}}
\setbeamercolor{item projected}{bg=ceruleanblue,fg=amber}
\setbeamertemplate{enumerate items}{%
\usebeamercolor[bg]{item projected}%
\raisebox{1.5pt}{\colorbox{bg}{\color{fg}\footnotesize\insertenumlabel}}%
}
\begin{enumerate}[<+->]
\item Métodos simples.
\item Métodos implícitos y de multipasos.
\item Métodos de Runge-Kutta.
\item Estabilidad de las soluciones.
\end{enumerate}
\end{frame}

\subsection*{Tema 4}

\begin{frame}
\frametitle{\textbf{Tema 4: Álgebra matricial}}
\setbeamercolor{item projected}{bg=ceruleanblue,fg=amber}
\setbeamertemplate{enumerate items}{%
\usebeamercolor[bg]{item projected}%
\raisebox{1.5pt}{\colorbox{bg}{\color{fg}\footnotesize\insertenumlabel}}%
}
\begin{enumerate}[<+->]
\item Inversión de matrices y número de condición.
\item Valores propios de matrices tridiagonales.
\item Discretización de la ecuación de Laplace y métodos iterativos de solución.
\item Solución numérica de ecuaciones diferenciales elípticas en una y dos dimensiones.
\end{enumerate}
\end{frame}

\subsection*{Tema 5}

\begin{frame}
\frametitle{\textbf{Tema 5: Métodos Monte Carlo}}
\setbeamercolor{item projected}{bg=ceruleanblue,fg=amber}
\setbeamertemplate{enumerate items}{%
\usebeamercolor[bg]{item projected}%
\raisebox{1.5pt}{\colorbox{bg}{\color{fg}\footnotesize\insertenumlabel}}%
}
\begin{enumerate}[<+->]
\item Números aleatorios.
\item Método de Monte Carlo.
\item Algoritmos de muestreo.
\end{enumerate}
\end{frame}

\subsection*{Tema 6}

\begin{frame}
\frametitle{\textbf{Tema 6: Ecuaciones diferenciales parciales}}
\setbeamercolor{item projected}{bg=ceruleanblue,fg=amber}
\setbeamertemplate{enumerate items}{%
\usebeamercolor[bg]{item projected}%
\raisebox{1.5pt}{\colorbox{bg}{\color{fg}\footnotesize\insertenumlabel}}%
}
\begin{enumerate}[<+->]
\item La ecuación de ondas y su discretización en diferencias finitas. Criterio de Courant.
\item La ecuación de Fourier para el calor y su discretización en diferencias finitas. Estabilidad del esquema.
\end{enumerate}
\end{frame}
\begin{frame}
\frametitle{Sobre el contenido del curso}
Como se ha revisado en el contenido de cada tema, vamos a considerar que ya cuentan con una formación correspondiente al sexto semestre de la carrera de Física.
\\
\bigskip
\pause
Por lo que en cada ejercicio o problema, el análisis y formalización de una expresión matemática se dará por hecho, así como su correspondiente solución analítica.
\end{frame}
\begin{frame}
\frametitle{Sobre el contenido del curso}
El alumno deberá de corroborar el planteamiento para el abordaje y solución de un problema de la física, considerando que está cursando el séptimo semestre de la carrera.
\end{frame}

\section{Evaluación del curso}
\frame{\tableofcontents[currentsection, hideothersubsections]}
\subsection{Evaluación}

\begin{frame}
\frametitle{Evaluación - Ejercicios}
Los elementos y la proporción de la calificación total del curso, se distribuyen de la siguiente manera:
\setbeamercolor{item projected}{bg=blue,fg=blond}
\setbeamertemplate{enumerate items}{%
\usebeamercolor[bg]{item projected}%
\raisebox{1.5pt}{\colorbox{bg}{\color{fg}\footnotesize\insertenumlabel}}%
}
\begin{enumerate}[<+->]
\item \textbf{Ejercicios en clase $\mathbf{10\%}$:} Durante la clase se trabajarán ejercicios, algunos de ellos se dejarán para que completen la solución, de tal forma que deberán de entregarlo resuelto para la siguiente sesión.
\\
\bigskip
\pause
\seti
\end{enumerate}
\end{frame}
\begin{frame}
\frametitle{Ejercicios en clase $\mathbf{10\%}$}
Durante la clase se trabajarán ejercicios, algunos de ellos se dejarán para que completen la solución, de tal forma que deberán de entregarlo resuelto para la siguiente sesión.
\end{frame}
\begin{frame}
\frametitle{Ejercicios en clase $\mathbf{10\%}$}
Para que el ejercicio resuelto se considere dentro de este porcentaje, se requiere que el alumno asista a la clase, en caso de que el alumno no asista y se entere del ejercicio, solamente se le revisará el ejercicio que entregue, pero no se le tomará en cuenta para el porcentaje, (moraleja: hay que asistir a clase).
\end{frame}
\begin{frame}
\frametitle{Evaluación - Tarea}
\setbeamercolor{item projected}{bg=blue,fg=blond}
\setbeamertemplate{enumerate items}{%
\usebeamercolor[bg]{item projected}%
\raisebox{1.5pt}{\colorbox{bg}{\color{fg}\footnotesize\insertenumlabel}}%
}
\begin{enumerate}[<+->]    
\conti
\item \textbf{Tarea $\mathbf{50\%}$} : Se tendrán tres tareas durante el curso, se les proporcionará de manera adelantada y con fecha de entrega definida, no se recibirán entregas extemporáneas.
\seti
\end{enumerate}
\end{frame}
\begin{frame}
\frametitle{De las Tareas}
Para que la tarea se considere, se deberá de entregar el $100\%$ de los ejercicios resueltos.
\\
\bigskip
\pause
En caso contrario, sólo se revisarán los ejercicios, pero no se tomará en cuenta como parte de la calificación por tareas.
\end{frame}
% \begin{frame}
% \frametitle{Del Examen - Tarea}
% Un examen-tarea se considera acreditado cuando la calificación obtenida es mayor o igual a seis.
% \\
% \bigskip
% \pause
% En caso de que en alguno (o más) examen-tarea, la calificación sea menor a seis, ya se es candidato a presentar el examen final del curso.
% \end{frame}
\begin{frame}
\frametitle{Evaluación - Examen}
\setbeamercolor{item projected}{bg=blue,fg=blond}
\setbeamertemplate{enumerate items}{%
\usebeamercolor[bg]{item projected}%
\raisebox{1.5pt}{\colorbox{bg}{\color{fg}\footnotesize\insertenumlabel}}%
}
\begin{enumerate}[<+->]    
\conti
\item \textbf{Examen $\mathbf{40\%}$} : Habrá tres exámenes en clase, de tipo teórico-prácticos. 
\item Se indicará oportunamente el día del examen y los temas correspondientes, que se resolverán y entregarán durante la clase.
\end{enumerate}
\end{frame}

\begin{frame}
\frametitle{Trabajo en equipo}
Podrán reunirse y colaborar para discutir, debatir, proponer y bosquejar la solución a los ejercicios de las tareas.
\\
\bigskip
En el dado caso de encontrar códigos idénticos, se cancelarán no sólo los ejercicios tipo copy-paste, sino la tarea completa del(los) alumnos involucrados.
\end{frame}

\begin{frame}
\frametitle{Calificación para aprobar el curso}
A partir de los elementos para la evaluación, se sumarán las calificaciones obtenidas, en caso de contar con un promedio final aprobatorio  es decir, una calificación mayor o igual a $6$ (seis), \pause esa calificación será la que se asiente en el acta del curso.
\end{frame}
\begin{frame}
\frametitle{Examen final}
Para presentar el examen final se deben de cumplir cada uno de los siguientes puntos:
\setbeamercolor{item projected}{bg=darkgreen,fg=darktangerine}
\setbeamertemplate{enumerate items}{%
\usebeamercolor[bg]{item projected}%
\raisebox{1.5pt}{\colorbox{bg}{\color{fg}\footnotesize\insertenumlabel}}%
}
\begin{enumerate}[<+->]
\item Que en un examen (o más), la calificación sea menor a seis. %Si los examen-tarea tienen calificación aprobatoria, no se permite presentar el examen final para \enquote{subir} la calificación del curso.
\item Haber entregado los tres exámenes parciales.
% \item Haber entregado el proyecto final.
\end{enumerate}
\end{frame}
\begin{frame}
\frametitle{Aplicación del examen final}
De acuerdo al Reglamento de Estudios Profesionales, habrá dos oportunidades para presentar el examen final, cuyas fechas se indican en el calendario del semestre 2024-1.
\end{frame}
\begin{frame}
\frametitle{Puntalizando sobre el examen final}
\setbeamercolor{item projected}{bg=falured,fg=laserlemon}
\setbeamertemplate{enumerate items}{%
\usebeamercolor[bg]{item projected}%
\raisebox{1.5pt}{\colorbox{bg}{\color{fg}\footnotesize\insertenumlabel}}%
}
\begin{enumerate}[<+->]
\item Si en la primera ronda de examen final, la calificación obtenida es aprobatoria (mayor o igual a seis), ésta es la que se asentará en el acta del curso, ya no se promedia con los otros elementos de evaluación.
\seti
\end{enumerate}
\end{frame}
\begin{frame}
\frametitle{Puntalizando sobre el examen final}
\setbeamercolor{item projected}{bg=falured,fg=laserlemon}
\setbeamertemplate{enumerate items}{%
\usebeamercolor[bg]{item projected}%
\raisebox{1.5pt}{\colorbox{bg}{\color{fg}\footnotesize\insertenumlabel}}%
}
\begin{enumerate}[<+->]
\conti    
\item Si la calificación del examen final en la primera ronda es no aprobatoria, se aplicará nuevamente un examen final en la segunda ronda. 
\\
\bigskip
\pause
La calificación obtenida en esta segunda ronda, es la que se asentará en el acta del curso.
\seti
\end{enumerate}
\end{frame}
\begin{frame}
\frametitle{Puntalizando sobre el examen final}
\setbeamercolor{item projected}{bg=falured,fg=laserlemon}
\setbeamertemplate{enumerate items}{%
\usebeamercolor[bg]{item projected}%
\raisebox{1.5pt}{\colorbox{bg}{\color{fg}\footnotesize\insertenumlabel}}%
}
\begin{enumerate}[<+->]
\conti    
\item Si el alumno no se presenta a la primera ronda del examen final, tendrá cinco como calificación final. Ya no podrá presentar la segunda ronda del examen final.
\end{enumerate}
\end{frame}
\begin{frame}
\frametitle{\textbf{¿En qué caso tendría NP o 5?}}
\emph{En el caso de haber presentado al menos un examen parcial y/o haber entregado al menos un ejercicio}, \pause pero si ya no se tiene un posterior registro de entregas, se considera que abandonaron el curso, al no cumplir con los puntos de la lista para el examen final, no se podrá presentar el examen final del curso.
\\
\bigskip
\pause
La calificación que se asentará en el acta final del curso será \textcolor{lava}{cinco (5)}.
\end{frame}
\begin{frame}
\frametitle{\textbf{¿En qué caso tendría NP o 5?}}
Se asentará en el acta de calificaciones \textcolor{blue}{No Presentó (NP)}, si y solo si: el alumno no entrega ejercicio alguno y no entrega algún examen-tarea (¿?).
\\
\bigskip
\pause
Ocupando nuevamente el Reglamento de Estudios Profesionales, tomen en cuenta que:
\pause
\begin{itemize}[<+->]
\item[\ding{212}] No \enquote{se guardan calificaciones}.
\item[\ding{212}] No se renuncia a una calificación.
\end{itemize}
\end{frame}

\subsection{Fechas importantes}

\begin{frame}
\frametitle{Fechas importantes}
\setbeamercolor{item projected}{bg=mediumblue,fg=magicmint}
\setbeamertemplate{enumerate items}{%
\usebeamercolor[bg]{item projected}%
\raisebox{1.5pt}{\colorbox{bg}{\color{fg}\footnotesize\insertenumlabel}}%
}
\begin{enumerate}[<+->]
\item Lunes 14 de agosto. Inicio del semestre 2024-1.
\item \textcolor{red}{Miércoles 1 de noviembre. Día feriado.}
\item \textcolor{red}{Lunes 20 de noviembre. Día feriado.}
\item Viernes 24 de noviembre. Fin de semestre 2024-1.
\item Del lunes 27 de noviembre al viernes 1 de diciembre, primera semana de finales.
\item Del lunes 4 al viernes 8 de diciembre, segunda semana de finales.
\end{enumerate}
\end{frame}

\end{document}