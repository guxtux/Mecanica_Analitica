\usepackage{listings}
\usepackage{xcolor}
\usepackage{textcomp}
\usepackage{color}
\definecolor{deepblue}{rgb}{0,0,0.5}
\definecolor{brown}{rgb}{0.59, 0.29, 0.0}
\definecolor{OliveGreen}{rgb}{0,0.25,0}
% \usepackage{minted}

% \DeclareCaptionFont{white}{\color{white}}
% \DeclareCaptionFormat{listing}{\colorbox{gray}{\parbox{0.98\textwidth}{#1#2#3}}}
% \captionsetup[lstlisting]{format=listing,labelfont=white,textfont=white}
\renewcommand{\lstlistingname}{Código}


\definecolor{Code}{rgb}{0,0,0}
\definecolor{Keywords}{rgb}{255,0,0}
\definecolor{Strings}{rgb}{255,0,255}
\definecolor{Comments}{rgb}{0,0,255}
\definecolor{Numbers}{rgb}{255,128,0}



\lstset{ 
language=Python,                % choose the language of the code
basicstyle=\normalsize\ttfamily,       % the size of the fonts that are used for the code
numbers=left,                   % where to put the line-numbers
numberstyle=\scriptsize,      % the size of the fonts that are used for the line-numbers
stepnumber=1,                   % the step between two line-numbers. If it is 1 each line will be numbered
numbersep=5pt,                  % how far the line-numbers are from the code
backgroundcolor=\color{white},  % choose the background color. You must add \usepackage{color}
showspaces=false,               % show spaces adding particular underscores
showstringspaces=false,         % underline spaces within strings
showtabs=false,                 % show tabs within strings adding particular underscores
frame=single,   		% adds a frame around the code
tabsize=2,  		% sets default tabsize to 2 spaces
captionpos=t,   		% sets the caption-position to bottom
breaklines=true,    	% sets automatic line breaking
breakatwhitespace=false,    % sets if automatic breaks should only happen at whitespace
escapeinside={\#},  % if you want to add a comment within your code
stringstyle =\color{OliveGreen},
%otherkeywords={{as}},             % Add keywords here
keywordstyle = \color{blue},
commentstyle = \color{black},
identifierstyle = \color{black},
literate=%
         {á}{{\'a}}1
         {é}{{\'e}}1
         {í}{{\'i}}1
         {ó}{{\'o}}1
         {ú}{{\'u}}1
%
%keywordstyle=\ttb\color{deepblue}
%fancyvrb = true,
}

\lstdefinestyle{FormattedNumber}{%
    literate={0}{{\textcolor{red}{0}}}{1}%
             {1}{{\textcolor{red}{1}}}{1}%
             {2}{{\textcolor{red}{2}}}{1}%
             {3}{{\textcolor{red}{3}}}{1}%
             {4}{{\textcolor{red}{4}}}{1}%
             {5}{{\textcolor{red}{5}}}{1}%
             {6}{{\textcolor{red}{6}}}{1}%
             {7}{{\textcolor{red}{7}}}{1}%
             {8}{{\textcolor{red}{8}}}{1}%
             {9}{{\textcolor{red}{9}}}{1}%
             {.0}{{\textcolor{red}{.0}}}{2}% Following is to ensure that only periods
             {.1}{{\textcolor{red}{.1}}}{2}% followed by a digit are changed.
             {.2}{{\textcolor{red}{.2}}}{2}%
             {.3}{{\textcolor{red}{.3}}}{2}%
             {.4}{{\textcolor{red}{.4}}}{2}%
             {.5}{{\textcolor{red}{.5}}}{2}%
             {.6}{{\textcolor{red}{.6}}}{2}%
             {.7}{{\textcolor{red}{.7}}}{2}%
             {.8}{{\textcolor{red}{.8}}}{2}%
             {.9}{{\textcolor{red}{.9}}}{2}%
             {\ }{{ }}{1}% handle the space
         ,%
          %mathescape=true
          escapeinside={__}
}
