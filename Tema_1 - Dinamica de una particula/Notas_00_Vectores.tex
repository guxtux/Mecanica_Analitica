\documentclass[12pt]{article}
\usepackage[utf8]{inputenc}
\usepackage[spanish,es-lcroman]{babel}
\usepackage{amsmath}
\usepackage{amsthm}
\usepackage{physics}
\usepackage{multicol,multienum}
\usepackage{graphicx}
\usepackage{float}
\usepackage{tikz}
\usetikzlibrary{positioning}
\usepackage{tikz-3dplot}
\usepackage{color}
\usepackage{anysize}
\usepackage{anyfontsize}
\usepackage{geometry}
\usepackage[autostyle,spanish=mexican]{csquotes}
\usepackage[shortlabels]{enumitem}
\usepackage[os=win]{menukeys}
\usepackage{bm}
\usepackage{calc}
%Este paquete permite manejar los encabezados del documento
\usepackage{fancyhdr}
%hay que definir el ambiente de la página
\pagestyle{fancy}
%aqui va el texto para todas las paginas l--> izquierda, r--> derecha, hay un C--> para centrar el texto deseado
%\lhead{Curso de Física Computacional}
\setlength{\headheight}{15pt}
\fancyhead[R]{\nouppercase{\leftmark}}
%define el ancho de la linea que separa el encabezado del cuerpo del texto
\renewcommand{\headrulewidth}{0.5pt}
\setlength{\parskip}{1em}
\renewcommand{\baselinestretch}{1.5}
\interfootnotelinepenalty=8000
\usepackage{hyperref}
%esta parte define el color del marco que aparece en las hiperreferencias.
\definecolor{links}{HTML}{2A1B81}
\hypersetup{colorlinks,linkcolor=,urlcolor=links}
\spanishdecimal{.}
\marginsize{1.5cm}{1.5cm}{1.5cm}{1.5cm}

\newlength{\depthofsumsign}
\setlength{\depthofsumsign}{\depthof{$\sum$}}
\newcommand{\nsum}[1][1.4]{% only for \displaystyle
    \mathop{%
        \raisebox
            {-#1\depthofsumsign+1\depthofsumsign}
            {\scalebox
                {#1}
                {$\displaystyle\sum$}%
            }
    }
}

\tikzset{>=latex}

\tikzset{
    ejesxy/.pic={
        \draw (0, 0) -- (5, 0);
        \draw (0, 0) -- (0, 4);
    }
}

\title{Vectores \\ \begin{Large} Curso de Mecánica Analítica \end{Large}}
\author{M. en C. Gustavo Contreras Mayén.}
\date{ }

\begin{document}
\maketitle
\fontsize{14}{14}\selectfont

\tableofcontents

\newpage

\section{Vectores.}

La Física es una ciencia que estudia cantidades físicas observables (medibles cuantitativamente) y las relaciones que pueden existir entre sus valores determinados experimentalmente. Algunas cantidades físicas requieren únicamente de la determinación de un solo número para su completa especificación. Temperatura, volumen, tiempo, longitud de recorrido, velocidad de la luz, frecuencia de una onda de sonido y carga eléctrica, son ejemplos de tales cantidades físicas, que son denominadas escalares. Por otra parte, hay cantidades físicas que requieren de dos o más números para su completa especificación. Para especificar un desplazamiento rectilíneo, por ejemplo, no es suficiente medir únicamente la magnitud del desplazamiento, sino que, para especificarlo en el espacio tridimensional, nos hacen falta dos cantidades más, con las que podremos determinar la dirección y sentido en que tiene lugar el desplazamiento. Cualquier cantidad física que, como el desplazamiento rectilíneo, requiera de una magnitud, una dirección y un sentido para su especificación, y que, además, pueda ser añadida a otra cantidad física similar, del mismo modo que son añadidos dos 
desplazamientos consecutivos para formar una sola cantidad física, también con una magnitud, una dirección y un sentido, recibe el nombre de cantidad vectorial tridimensional o simplemente vector tridimensional. En este capítulo familiarizaremos al lector con el lenguaje matemático y la manipulación matemática fundamental de los vectores tridimensionales. En uno de los últimos capítulos generalizaremos nuestros resultados para vectores de $n$ dimensiones, que son cantidades que requieren de $n$ números para su especificación. 
\par
Usaremos las letras negritas para señalar los vectores y distinguirlos de los escalares, que estarán indicados por letras cursivas. Asi, $\vb{A}$ representa un vector, mientras que su magnitud, siendo un número puro o escalar, será representada por $A$. En ocasiones, también utilizaremos $\abs{A}$ para representar la magnitud $A$ del vector $\vb{A}$.

\subsection{Representación geométrica de un vector.} 

Gráficamente podemos representar la dirección, sentido y magnitud de un vector por un segmento de recta dirigido, o sea, con una flecha trazada paralelamente al vector y apuntando en el sentido de éste. La longitud de la flecha será dibujada a escala, de manera que esta longitud pueda representar la magnitud del vector. En la figura (\ref{fig:figura_01_01}), por ejemplo, representamos gráficamente un vector $\vb{A}$, paralelo al plano $x y$ y que forma un ángulo $\phi$ con una recta paralela al eje $x$. Si el vector $\vb{A}$ tiene una magnitud de cinco unidades, entonces, la longitud de la flecha que lo representa se dibujaría igual a cinco veces la longitud escogida para representar al vector de magnitud unidad. 
\begin{figure}[H]
    \centering
    \begin{tikzpicture}
        \draw (0, 0) -- (5, 0) node[below, pos=1.05] {$x$};
        \draw (0, 0) -- (0, 4) node[left, pos=1.05] {$y$};
        \draw [-stealth, thick] (1, 1) -- (3, 3) node[above, midway] {$\vb{A}$};
        \draw [dashed] (1, 1) -- (3, 1);
        \draw (1.5, 1) arc(0:30:0.75) node[right] {$\phi$};
    \end{tikzpicture}
    \caption{Representación gráfica de un vector $\vb{A}$ en el plano $x y$.}
    \label{fig:figura_01_01}
\end{figure}
De la representación de la magnitud de un vector por la longitud de una flecha deducimos que la multiplicación de un vector $\vb{A}$ por un escalar $c$ (positivo) da como producto un vector paralelo a $\vb{A}$, que tiene una magnitud igual a $c$ veces la magnitud de $\vb{A}$:
\begin{align}
    \abs{c \, \vb{A}} = c \, A
    \label{eq:ecuacion_01_01}
\end{align} 
De esto, resulta que podríamos expresar cualquier vector como el producto de un escalar por un vector de magnitud uno, llamado vector unidad. Está claro que si $\vb{e}_{A}$ es el vector unidad en la dirección y sentido de $\vb{A}$, el vector $A \, \vb{e}_{A}$ es un vector en la dirección y sentido de $\vb{A}$ que tiene una magnitud $A$. 
\par
Se dice que dos vectores $\vb{A}$ y $\vb{B}$ son iguales si ambos tienen las mismas magnitud y dirección y el mismo sentido. Por lo tanto, podemos escribir:
\begin{align}
    \vb{B} = A \, \vb{e}_{A}
    \label{eq:ecuacion_01_02} 
\end{align}
El vector nulo $\vb{0}$ se define como un vector de magnitud igual a cero y, por lo tanto, no tiene dirección ni sentido.

\subsection{Suma y resta de vectores.}

La suma de dos vectores $\vb{A}$ y $\vb{B}$ está representada gráficamente en la figura (\ref{fig:figura_01_02}). Para sumar gráficamente dos vectores $\vb{A}$ y $\vb{B}$, dibujamos la flecha que representa al vector $\vb{A}$, y desde su punta trazaremos la flecha que representa al vector $\vb{B}$. El vector $\vb{C}$, que es la suma de los dos vectores, está entonces representado por la flecha dibujada desde la cola de la flecha de $\vb{A}$, hasta la punta de la de $\vb{B}$. Esta regla para la suma de vectores se llama \emph{ley de adición del paralelogramo}.
\begin{figure}[H]
    \centering
    \begin{tikzpicture}
        \draw (0, 0) -- (5, 0);
        \draw (0, 0) -- (0, 4);
        \draw [-stealth, thick] (1, 1) -- (3, 3);
        \node[rotate=45] at (1.7, 2.25) {$\vb{C} = \vb{A} + \vb{B}$};
        \draw [-stealth, thick] (1, 1) -- (3, 1) node [below, midway] {$\vb{A}$};
        \draw [-stealth, thick] (3, 1) -- (3, 3) node [right, midway] {$\vb{B}$};
    \end{tikzpicture}
    \caption{Representación gráfica de la suma de dos vectores.}
    \label{fig:figura_01_02}
\end{figure}
En la figura (\ref{fig:figura_01_03}) podemos ver que la suma de vectores es conmutativa:
\begin{align*}
    \vb{A} + \vb{B} = \vb{B} + \vb{A}
\end{align*}
\begin{figure}[H]
    \centering
    \begin{tikzpicture}[inner sep=0pt]
        \node at (0, 0) (A) {};
        \node at (4, 1) (B) {};
        \node at (4, 4) (C) {};
        \node at (0, 3) (D) {};
        \draw [-stealth, thick] (A) -- (B) node [below=0.25, midway] {$\vb{A}$};
        \draw [-stealth, thick] (B) -- (C) node [right=0.25, midway] {$\vb{B}$};
        \draw [-stealth, thick] (A) -- (C);
        \node at (1.5, 2) [rotate=45] {$\vb{C} = \vb{A} + \vb{B}$};
        \draw [-stealth, thick] (A) -- (D) node [left=0.25, midway] {$\vb{B}$};
        \draw [-stealth, thick] (D) -- (C) node [above=0.25, midway] {$\vb{A}$};
    \end{tikzpicture}
    \caption{Naturaleza conmutativa de la suma vectorial: $\vb{A} + \vb{B} = \vb{B} + \vb{A}$.}
    \label{fig:figura_01_03}
\end{figure}
La suma de vectores es también asociativa, ver la figura (\ref{fig:figura_01_04}):
\begin{align*}
    \vb{A} + \left( \vb{B} + \vb{C} \right) = \left( \vb{A} + \vb{B} \right) + \vb{C}
\end{align*}
\begin{figure}[H]
    \centering
    \begin{tikzpicture}[inner sep=0pt]
        \node at (0, 0) (A) {};
        \node at (4, 1) (B) {};
        \node at (5, 4) (C) {};
        \node at (2.5, 6) (D) {};
        \draw [-stealth, thick] (A) -- (B) node [below=0.25, midway] {$\vb{A}$};
        \draw [-stealth, thick] (B) -- (C) node [right=0.25, midway] {$\vb{B}$};
        \draw [-stealth, thick] (A) -- (C);
        \node at (1.75, 1.75) [rotate=40] {$\vb{A} + \vb{B}$};
        \draw [-stealth, thick] (C) -- (D) node [above=0.25, midway] {$\vb{C}$};
        % \draw [-stealth, thick] (A) -- (D) node [left=0.25, midway] {$\vb{B}$};
        % \draw [-stealth, thick] (D) -- (C) node [above=0.25, midway] {$\vb{A}$};
        \draw [-stealth, thick] (B) -- (D) node [left=0.23, pos=0.5] {$\vb{B} + \vb{C}$};
        \draw [-stealth, thick] (A) -- (D);
        \node at (0.75, 3) [rotate=68] {$\vb{A} + \vb{B} + \vb{C}$};
    \end{tikzpicture}
    \caption{Naturaleza asociativa de la suma vectorial : $\vb{A} + \left( \vb{B} + \vb{C} \right) = \left( \vb{A} + \vb{B} \right) + \vb{C}$.}
    \label{fig:figura_01_04}
\end{figure}
Si la suma de dos vectores $\vb{A}$ y $\vb{B}$ es igual a un vector dé magnitud cero:
\begin{align}
    \vb{A} + \vb{B} = \vb{0}
    \label{eq:ecuacion_01_03}
\end{align}
entonces, los dos vectores $\vb{A}$ y $\vb{B}$, obviamente, deben tener iguales magnitud y dirección y sentidos opuestos. Asi, la ecuación (\ref{eq:ecuacion_01_03}) queda :
\begin{align}
    \vb{B} = - \vb{A}
    \label{ec:ecuacion_01_04}
\end{align}
que nos dice que $- \vb{A}$ es un vector que tiene la misma magnitud (y dirección) del vector $\vb{A}$, pero apunta en un sentido opuesto al de éste. 
\par
Esto nos permite definir la resta del vector $\vb{B}$ al vector $\vb{A}$, como la suma del vector $\vb{A}$ y el vector $\left( - \vb{B} \right)$: 
\begin{align}
    \vb{A} - \vb{B} = \vb{A} + \left( - \vb{B} \right)
    \label{eq:ecuacion_01_05}
\end{align}

\subsection{Representación algebraica de un vector.}

Cualquier vector $\vb{A}$, como demostraremos, puede ser representado algebraicamente, especificando sus proyecciones sobre un sistema de ejes 
coordenados o de vectores base. Los tres vectores base de un sistema de coordenadas tridimensional deben ser independientes linealmente; por 
tanto, deben satisfacer el requisito de no ser coplanares. La elección más simple, aunque no queremos decir que sea la única, para tal sistema de vectores base no coplanares, es un sistema de vectores unidad perpendiculares entre sí. 
\par
En un sistema de coordenadas cartesianas se escogerán como los tres vectores base los vectores unidad tomados y dirigidos respectivamente a lo 
largo de los ejes $x$, $y$ y $z$ positivos. Los vectores unidad sobre los ejes coordenados $x$, $y$ y $z$ positivos se designarán, respectivamente, por los símbolos $\vb{i}$, $\vb{j}$ y $\vb{k}$ (figura \ref{fig:figura_01_05}). A veces, también encontraremos notacionalmente conveniente designarlos por los símbolos $\vb{e}_{1}, \vb{e}_{2}$ y $\vb{e}_{3}$.
\begin{figure}[H]
    \centering
    \tdplotsetmaincoords{70}{110}
    \begin{tikzpicture}[tdplot_main_coords, scale=2]
        \draw[thick] (0, 0, 0) -- (2, 0, 0) node[anchor=north east] {$x$};
        \draw[thick] (0, 0, 0) -- (0, 1.25, 0) node[anchor=north west] {$y$};
        \draw[thick] (0, 0, 0) -- (0, 0, 1) node[anchor=south] {$z$};
        \draw[line width=0.5mm, -stealth] (0, 0, 0) -- (1, 0, 0) node[above, pos=0.9] {$\vb{i}$};
        \draw[line width=0.5mm, -stealth] (0, 0, 0) -- (0, 0.5, 0) node[above, pos=0.9] {$\vb{j}$};
        \draw[line width=0.5mm, -stealth] (0, 0, 0) -- (0, 0, 0.5) node[right, pos=0.9] {$\vb{k}$};
    \end{tikzpicture}
    \caption{Vectores base unitarios cartesianos.}
    \label{fig:figura_01_05}
\end{figure}
La proyección de un vector $\vb{A}$ sobre otro $\vb{B}$ está representada gráficamente por la longitud del segmento rectilíneo comprendido entre la intersección de una recta paralela a $\vb{B}$ con las perpendiculares a la dirección de $\vb{B}$ desde la cola y la punta de la flecha que representa a $\vb{A}$ (figura \ref{fig:figura_01_06}).
\begin{figure}[H]
    \centering
    \begin{tikzpicture}
        \draw (-0.5, -0.5) -- (6, -0.5);
        \draw (-0.5, -0.5) -- (-0.5, 4);
        \draw [-stealth, thick] (1, 1) -- (4, 3) node[above=0.25, midway] {$\vb{A}$};
        \draw [dashed] (4, 3) -- (4, 1);
        \draw (1, 1) -- (5, 1);
        \draw (1.5, 1) arc(0:30:0.5) node[right] {$\phi$};
        \draw (1, 1) -- (1, 0.6);
        \draw (4, 1) -- (4, 0.6);
        \draw [-stealth] (1.3, 0.7) -- (1, 0.7);
        \draw [-stealth] (3.7, 0.7) -- (4, 0.7);
        \node at (2.5, 0.7) {\small{$A_{B} = A \cos \phi$}};
        \draw [-stealth, thick] (2.5, 0.15) -- (5.5, 0.15) node[below, midway] {$\vb{B}$};
    \end{tikzpicture}
    \caption{Proyección del vector $\vb{A}$ sobre el vector $\vb{B}$.}
    \label{fig:figura_01_06}
\end{figure}
Si la proyección sobre $\vb{B}$ de la punta de la flecha que representa a $\vb{A}$, se halla en el sentido de $\vb{B}$ con respecto a la proyección de la cola de dicha flecha sobre $\vb{B}$, entonces, la proyección será considerada positiva. Si está dirigida en sentido opuesto, será considerada negativa. En relación con el más pequeño de los dos ángulos, el ángulo $\phi$, que forma la flecha que representa al vector $\vb{A}$ con una flecha trazada en la dirección y sentido de $\vb{B}$, a partir del pie de la flecha de $\vb{A}$, la proyección de $\vb{A}$ sobre $\vb{B}$, también llamada componente $\vb{B}$ de $\vb{A}$ y designada por $A_{B}$ , está dada por la fórmula:
\begin{align}
    A_{B} = A \cos \phi
    \label{eq:ecuacion_01_06}
\end{align}
El ángulo $\phi$ será mencionado como el ángulo entre los vectores $\vb{A}$ y $\vb{B}$. 
\par
De acuerdo con la regla por la que indicamos que una proyección es positiva o negativa, encontramos que $A_{B}$ es positiva para:
\begin{align*}
    0 < \phi < \dfrac{\pi}{2}
\end{align*}
y negativa para:
\begin{align*}
    \dfrac{\pi}{2} < \phi < \pi
\end{align*}
De la ley de la suma se desprende que el vector $\vb{A}$ se puede expresar por la suma de tres vectores $A_{x} \, \vb{i}, A_{y} \, \vb{j}$ y $A_{z} \, \vb{k}$. Esto es:
\begin{align}
    \vb{A} = A_{x} \, \vb{i} + A_{y} \, \vb{j} + A_{z} \, \vb{k}
    \label{eq:ecuacion_01_07}
\end{align}
donde $A_{x}, A_{y}$ y $A_{z}$ son las componentes de $\vb{A}$ sobre los ejes $x, y, z$ positivos. Por ejemplo, considérese un vector $\vb{A}$ que esté en el plano $x y$ $(A_{z} = 0)$. En este caso, siempre podremos imaginar al vector $\vb{A}$ situado sobre la hipotenusa de un triángulo rectángulo cuyos catetos sean paralelos a los ejes $x$ y $y$ positivos. De la figura (\ref{fig:figura_01_07}) resulta evidente que:
\begin{align}
    A_{x} &= A \cos \phi \label{eq:ecuacion_01_08} \\
    A_{y} &= A \sin \phi \label{eq:ecuacion_01_09}
\end{align}
y que la relación:
\begin{align}
    A^{2} &= A_{x}^{2} + A_{y}^{2} \label{eq:ecuacion_01_10} \\
    \vb{A} &= A_{x} \, \vb{i} + A_{y} \, \vb{j} \label{eq:ecuacion_01_11}
\end{align}
es correcta.
\begin{figure}[H]
    \centering
    \begin{tikzpicture}
        \pic at (0, 0) {ejesxy};
        \draw [-stealth, thick] (0, 0) -- (3, 3) node[above=0.25, midway] {$\vb{A}$};
        \draw [-stealth, thick] (0, 0) -- (3, 0) node[below, midway] {$A_{x} \, \vb{i}$};
        \draw [-stealth, thick] (3, 0) -- (3, 3) node[right, midway] {$A_{y} \, \vb{j}$};
        \draw (0.5, 0) arc(0:45:0.5) node[right] {$\phi$};
    \end{tikzpicture}
    \caption{Prueba gráfica de $\vb{A} = A_{x} \, \vb{i} + A_{y} \, \vb{j}$ para un vector paralelo al plano $x y$.}
    \label{fig:figura_01_07}
\end{figure}
La extensión a vectores tridimensionales se muestra en la figura (\ref{fig:figura_01_08}). En relación con el ángulo $\theta$ que forma el vector $\vb{A}$ con el eje $z$ positivo y el ángulo $\phi$ que forma la proyección de $\vb{A}$ sobre el plano $x y$ con el eje $x$ positivo, resulta que:
\begin{align}
    A_{x} &= A \sin \theta \cos \phi \label{eq:ecuacion_01_12} \\[0.5em]
    A_{y} &= A \sin \theta \sin \phi \label{eq:ecuacion_01_13} \\[0.5em]
    A_{z} &= A \cos \theta \label{eq:ecuacion_01_14} \\[0.5em]
    A^{2} &= A_{x}^{2} + A_{y}^{2} + A_{z}^{2} \label{eq:ecuacion_01_15}
\end{align}
y que realmente:
\begin{align*}
    \vb{A} &= A_{x} \, \vb{i} + A_{y} \, \vb{j} + A_{z} \, \vb{k}
\end{align*}
\begin{figure}[H]
    \centering
    \tdplotsetmaincoords{60}{120}
\begin{tikzpicture}
	[scale=3,
		tdplot_main_coords,
		axis/.style={-stealth, black, thick},
		vector/.style={-stealth, black, thick},
		vector guide/.style={blue, thick},
		angle/.style={red,thick}]

	%standard tikz coordinate definition using x, y, z coords
	\coordinate (O) at (0, 0, 0);
	
	%tikz-3dplot coordinate definition using r, theta, phi coords
	\tdplotsetcoord{P}{1.5}{55}{60}
	
	%draw axes
	\draw[axis] (0, 0, 0) -- (1.5, 0, 0) node [anchor=north east] {$x$};
	\draw[axis] (0, 0, 0) -- (0, 1, 0) node [anchor=north west] {$y$};
	\draw[axis] (0, 0, 0) -- (0, 0, 1) node [anchor=south] {$z$};
	
	%draw a vector from O to P
	\draw[vector] (O) -- (P) node [above, pos=0.7] {$\vb{A}$};
	
	%draw guide lines to components
	\draw[vector guide] (O) -- (Pxy);
	\draw[vector guide, -stealth] (Pxy) -- (P) node [right, pos=0.6] {$A_{z} \, \vb{k}$};


    \draw [-stealth, vector guide] (O) -- (Px) node [left, pos=0.5] {$A_{x} \, \vb{i}$};
    \draw [-stealth, vector guide] (Px) -- (Pxy) node [below, pos=0.75] {$A_{y} \, \vb{j}$};
    \node at (1, 0.35, 0) [blue, below] {$A_{x} \, \vb{i} + A_{y} \, \vb{j}$};
    \draw [blue, thick, -stealth] (1, 0.35, 0) -- (0.4, 0.65, 0);

	%draw an arc illustrating the angle defining the orientation
	\tdplotdrawarc[angle]{(O)}{.25}{0}{60}{anchor=north}{$\phi$}

	%define the rotated coordinate frame to lie in the "theta plane"
	\tdplotsetthetaplanecoords{55}
	
	\tdplotdrawarc[tdplot_rotated_coords,angle]{(O)}{.35}{0}{55}
          {anchor=south west}{$\theta$}

    \end{tikzpicture}
    \caption{Prueba gráfica de $\vb{A} = A_{x} \, \vb{i} + A_{y} \, \vb{j} + A_{z} \, \vb{k}$}
    \label{fig:figura_01_08}
\end{figure}
De la definición de igualdad de dos vectores resultará evidente que la de sus componentes $x, y, z$ es una condición necesaria y suficiente para que sean iguales. También sera claro que las componentes $x, y, z$ de la suma de dos o mas vectores son iguales a las sumas de las respectivas componentes de los vectores. Esto es, si el vector $\vb{C}$ es la suma de los vectores $\vb{A}$ y $\vb{B}$: 
\begin{align}
    \vb{C} = \vb{A} + \vb{B}
    \label{eq:ecuacion_01_16}
\end{align}
entonces:
\begin{align}
    C_{x} = A_{x} + B_{x}, \hspace{0.75cm} C_{y} = A_{y} + B_{y}, \hspace{0.75cm} C_{z} = A_{z} + B_{z} \label{eq:ecuacion_01_17}
\end{align}
de donde podemos obtener:
\begin{align}
    C^{2} = C_{x}^{2} + C_{y}^{2} + C_{z}^{2} &= A^{2} + B^{2} + 2 \left( A_{x} B_{x} + A_{y} B_{y} + A_{z} B_{z} \right) \nonumber \\[0.5em]
    &= A^{2} + B^{2} + 2 \, A \, B \, \cos \phi \label{eq:ecuacion_01_18}
\end{align}
La ultima expresión es la ley de los cosenos. Observaremos que el signo más frente al ultimo término del segundo miembro de la ecuación (\ref{eq:ecuacion_01_18}) se debe a que el ángulo entre los vectores $\vb{A}$ y $\vb{B}$ es el exterior del triángulo formado por estos vectores, ver la figura (\ref{fig:figura_01_09}). 
\begin{figure}[H]
    \centering
    \begin{tikzpicture}[inner sep=0pt]
        \node at (1, 0.5) (A) {};
        \node at (4, 1.5) (B) {};
        \node at (5, 4) (C) {};

        \pic at (0, 0) {ejesxy};

        \draw [-stealth, thick] (A) -- (B) node [below, midway] {$\vb{A}$};
        \draw [-stealth, thick] (B) -- (C) node [right=0.2, pos=0.7] {$\vb{B}$};
        \draw [-stealth, thick] (A) -- (C) node [above=0.2, midway] {$\vb{C}$};
        \draw (B) -- (5.5, 2);
        \draw (4.5, 1.7) arc(0:30:1.3) node[right=0.1, pos=0.5] {$\phi$};
    \end{tikzpicture}
    \caption{Ley de los cosenos $C^{2} = A^{2} + B^{2} + 2 \, A \, B \, \cos \phi$.}
    \label{fig:figura_01_09}
\end{figure}
La elección de la orientación del sistema de coordenadas $(x, y, z)$ es, por supuesto, arbitraria. Por lo tanto, dos vectores iguales tienen componentes iguales en cualquier dirección, y la componente en cualquier dirección de la suma de dos o más vectores es igual a la suma de las componentes de estos vectores en dicha dirección:
\begin{align}
    C_{D} = A_{D} + B_{D}
    \label{eq:ecuacion_01_19}
\end{align}

\subsection{Multiplicación vectorial.}

Dados los dos vectores $\vb{A}$ y $\vb{B}$, existen dos productos de estos vectores para los que encontraremos uso inmediato. El primer producto es el llamado \emph{producto escalar}, ya que nos da una cantidad escalar. Hemos hallado ya este producto en la ecuación (\ref{eq:ecuacion_01_18}). Se designa por $\vb{A} \cdot \vb{B}$, y está definido por la ecuación:
\begin{align}
    \vb{A} \cdot \vb{B} = A \, B \, \cos \phi
    \label{eq:ecuacion_01_20}
\end{align}
donde $\phi$ es el ángulo que forman los vectores $\vb{A}$ y $\vb{B}$. Como hemos visto, esta expresión aparece en la ley de los cosenos, de la que obtenemos la relación (fig. \ref{fig:figura_01_09}):
\begin{align}
    \vb{A} \cdot \vb{B} = \dfrac{C^{2} - A^{2} - B^{2}}{2} = A_{x} \, B_{x} + A_{y} \, B_{y} + A_{z} \, B_{z}
    \label{eq:ecuacion_01_21}
\end{align}
De su definición, el producto escalar es, obviamente, conmutativo:
\begin{align*}
    \vb{A} \cdot \vb{B} = \vb{B} \cdot \vb{A}
\end{align*}

\end{document}