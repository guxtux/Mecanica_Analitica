\documentclass[12pt]{beamer}
\usepackage{../Estilos/BeamerFC}
\usepackage{../Estilos/ColoresLatex}
% \usefonttheme{serif}
\input{../Preambulos/preambulo_Beamer_Copenhagen_wolverine}

\title{Leyes de Newton}
\subtitle{Tema 1 - Dinámica de una partícula}
% \date{}
% \author{M. en C. Gustavo Contreras Mayén.}


\newcommand\RBox[1]{%
  \tikz\node[draw,rounded corners,align=center,] {#1};%
}

\AtBeginDocument{\RenewCommandCopy\qty\SI}
\ExplSyntaxOn
\msg_redirect_name:nnn { siunitx } { physics-pkg } { none }
\ExplSyntaxOff

\resetcounteronoverlays{saveenumi}

\begin{document}
\fontsize{14}{14}\selectfont
\spanishdecimal{.}
\maketitle

\section*{Contenido}
\frame[allowframebreaks]{\frametitle{Contenido} \tableofcontents[currentsection, hideallsubsections]}

\section{Leyes de Newton}
\frame[allowframebreaks]{\frametitle{Temas a revisar} \tableofcontents[currentsection, hideothersubsections]}
\subsection{Primera revisión}

\begin{frame}
\frametitle{Objetivo de la mecánica}
La mecánica es una rama de la física que se ocupa del estudio del movimiento de los cuerpos y las fuerzas que actúan sobre ellos.
\end{frame}
\begin{frame}
\frametitle{Objetivo de la mecánica}
Su objetivo principal es describir, analizar y predecir el comportamiento de los sistemas físicos en función de leyes matemáticas.
\end{frame}
\begin{frame}
\frametitle{Objetivo de la mecánica}
Como herramientas matemáticas se considerarán entre otros:
\setbeamercolor{item projected}{bg=coquelicot,fg=white}
\setbeamertemplate{enumerate items}{%
\usebeamercolor[bg]{item projected}%
\raisebox{1.5pt}{\colorbox{bg}{\color{fg}\footnotesize\insertenumlabel}}%
}
\begin{enumerate}[<+->]
\item Ecuaciones diferenciales.
\item Matrices.
\item Cálculo vectorial.
\item Álgebra compleja.
\end{enumerate}
\end{frame}
\begin{frame}
\frametitle{Objetivo de la mecánica}
Estas leyes permiten entender fenómenos físicos que ocurren en nuestra vida diaria y en escalas macroscópicas.
\end{frame}
\begin{frame}
\frametitle{Alcance de la mecánica clásica}
La mecánica clásica no es adecuada para estudiar fenómenos que ocurren a velocidades cercanas a la velocidad de la luz (\textocolor{red}{relatividad}) \pause o a escala subatómica (\textocolor{ao}{mecánica cuántica}).
\end{frame}

\subsection{Conceptos básicos}

\begin{frame}
\frametitle{Partícula puntual}
Una \textocolor{cobalt}{partícula puntual}, que también se le conoce como masa puntual, punto material, o de manera más clara: \textocolor{carmine}{partícula}.
\end{frame}
\begin{frame}
\frametitle{Partícula puntual}
Es una idealización de un objeto cuando solo nos interesa el estudio de la posición de un punto de ese objeto, haciendo a un lado propiedades del mismo: extensión, color, volumen, etc.
\end{frame}

\end{document}