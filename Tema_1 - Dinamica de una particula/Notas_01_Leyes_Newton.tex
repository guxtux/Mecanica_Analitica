\documentclass[12pt]{beamer}
\usepackage{../Estilos/BeamerFC}
\usepackage{../Estilos/ColoresLatex}
% \usefonttheme{serif}
\usetheme{Copenhagen}
\usecolortheme{wolverine}
%\useoutertheme{default}
\setbeamercovered{invisible}
% or whatever (possibly just delete it)
\setbeamertemplate{section in toc}[sections numbered]
\setbeamertemplate{subsection in toc}[subsections numbered]
\setbeamertemplate{subsection in toc}{\leavevmode\leftskip=3.2em\rlap{\hskip-2em\inserttocsectionnumber.\inserttocsubsectionnumber}\inserttocsubsection\par}
% \setbeamercolor{section in toc}{fg=blue}
% \setbeamercolor{subsection in toc}{fg=blue}
% \setbeamercolor{frametitle}{fg=blue}
\setbeamertemplate{caption}[numbered]

\setbeamertemplate{footline}
\beamertemplatenavigationsymbolsempty
\setbeamertemplate{headline}{}


\makeatletter
% \setbeamercolor{section in foot}{bg=gray!30, fg=black!90!orange}
% \setbeamercolor{subsection in foot}{bg=blue!30}
% \setbeamercolor{date in foot}{bg=black}
\setbeamertemplate{footline}
{
  \leavevmode%
  \hbox{%
  \begin{beamercolorbox}[wd=.333333\paperwidth,ht=2.25ex,dp=1ex,center]{section in foot}%
    \usebeamerfont{section in foot} \insertsection
  \end{beamercolorbox}%
  \begin{beamercolorbox}[wd=.333333\paperwidth,ht=2.25ex,dp=1ex,center]{subsection in foot}%
    \usebeamerfont{subsection in foot}  \insertsubsection
  \end{beamercolorbox}%
  \begin{beamercolorbox}[wd=.333333\paperwidth,ht=2.25ex,dp=1ex,right]{date in head/foot}%
    \usebeamerfont{date in head/foot} \insertshortdate{} \hspace*{2em}
    \insertframenumber{} / \inserttotalframenumber \hspace*{2ex} 
  \end{beamercolorbox}}%
  \vskip0pt%
}
\makeatother

\makeatletter
\patchcmd{\beamer@sectionintoc}{\vskip1.5em}{\vskip0.8em}{}{}
\makeatother

% %\newlength{\depthofsumsign}
% \setlength{\depthofsumsign}{\depthof{$\sum$}}
% \newcommand{\nsum}[1][1.4]{% only for \displaystyle
%     \mathop{%
%         \raisebox
%             {-#1\depthofsumsign+1\depthofsumsign}
%             {\scalebox
%                 {#1}
%                 {$\displaystyle\sum$}%
%             }
%     }
% }
% \def\scaleint#1{\vcenter{\hbox{\scaleto[3ex]{\displaystyle\int}{#1}}}}
% \def\scaleoint#1{\vcenter{\hbox{\scaleto[3ex]{\displaystyle\oint}{#1}}}}
% \def\bs{\mkern-12mu}


\title{Leyes de Newton}
\subtitle{Tema 1 - Dinámica de una partícula}
% \date{}
% \author{M. en C. Gustavo Contreras Mayén.}


\newcommand\RBox[1]{%
  \tikz\node[draw,rounded corners,align=center,] {#1};%
}

\AtBeginDocument{\RenewCommandCopy\qty\SI}
\ExplSyntaxOn
\msg_redirect_name:nnn { siunitx } { physics-pkg } { none }
\ExplSyntaxOff

\resetcounteronoverlays{saveenumi}

\begin{document}
\fontsize{14}{14}\selectfont
\spanishdecimal{.}
\maketitle

\section*{Contenido}
\frame[allowframebreaks]{\frametitle{Contenido} \tableofcontents[currentsection, hideallsubsections]}

\section{Leyes de Newton}
\frame[allowframebreaks]{\frametitle{Temas a revisar} \tableofcontents[currentsection, hideothersubsections]}
\subsection{Primera revisión}

\begin{frame}
\frametitle{Objetivo de la mecánica}
La mecánica es una rama de la física que se ocupa del estudio del movimiento de los cuerpos y las fuerzas que actúan sobre ellos.
\end{frame}
\begin{frame}
\frametitle{Objetivo de la mecánica}
Su objetivo principal es describir, analizar y predecir el comportamiento de los sistemas físicos en función de leyes matemáticas.
\end{frame}
\begin{frame}
\frametitle{Objetivo de la mecánica}
Como herramientas matemáticas se considerarán entre otros:
\setbeamercolor{item projected}{bg=coquelicot,fg=white}
\setbeamertemplate{enumerate items}{%
\usebeamercolor[bg]{item projected}%
\raisebox{1.5pt}{\colorbox{bg}{\color{fg}\footnotesize\insertenumlabel}}%
}
\begin{enumerate}[<+->]
\item Ecuaciones diferenciales.
\item Matrices.
\item Cálculo vectorial.
\item Álgebra compleja.
\end{enumerate}
\end{frame}
\begin{frame}
\frametitle{Objetivo de la mecánica}
Estas leyes permiten entender fenómenos físicos que ocurren en nuestra vida diaria y en escalas macroscópicas.
\end{frame}
\begin{frame}
\frametitle{Alcance de la mecánica clásica}
La mecánica clásica no es adecuada para estudiar fenómenos que ocurren a velocidades cercanas a la velocidad de la luz (\textocolor{red}{relatividad}) \pause o a escala subatómica (\textocolor{ao}{mecánica cuántica}).
\end{frame}

\subsection{Conceptos básicos}

\begin{frame}
\frametitle{Partícula puntual}
Una \textocolor{cobalt}{partícula puntual}, que también se le conoce como masa puntual, punto material, o de manera más clara: \textocolor{carmine}{partícula}.
\end{frame}
\begin{frame}
\frametitle{Partícula puntual}
Es una idealización de un objeto cuando solo nos interesa el estudio de la posición de un punto de ese objeto, haciendo a un lado propiedades del mismo: extensión, color, volumen, etc.
\end{frame}

\end{document}